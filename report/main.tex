\documentclass[12pt]{article}

% Pakiety językowe i kodowanie
\usepackage[utf8]{inputenc}
\usepackage[T1]{fontenc}
\usepackage{polski}
\usepackage[english,polish]{babel}

% Geometria strony
\usepackage[margin=2.5cm]{geometry}

% Pakiety matematyczne
\usepackage{amsmath, amssymb, amsthm, mathtools}
\usepackage{bm} % Pogrubione symbole matematyczne

% Pakiety graficzne i wykresy
\usepackage{graphicx}
\usepackage{tikz}
\usetikzlibrary{matrix, positioning, arrows.meta, shapes.geometric, calc}
\usepackage{pgfplots}
\pgfplotsset{compat=1.17}

% Ulepszona typografia i renderowanie fontów
\usepackage{microtype}
\usepackage{lmodern} % Latin Modern fonts - lepsza jakość niż Computer Modern

% Algorytmy
\usepackage{algorithm}
\usepackage{algpseudocode}
\floatname{algorithm}{Algorytm}
\renewcommand{\listalgorithmname}{Spis algorytmów}
\renewcommand{\algorithmicrequire}{\textbf{Dane wejściowe:}}
\renewcommand{\algorithmicensure}{\textbf{Wynik:}}

% Linki i metadane
\usepackage{hyperref}
\hypersetup{
    colorlinks=true,
    linkcolor=black,
    filecolor=magenta,      
    urlcolor=blue,
    pdftitle={Dokumentacja Projektowa - Atak Primal},
    pdfauthor={Kryptologia Akademicka}
}

% Konfiguracja nagłówków
\usepackage{titlesec}
\titlelabel{\thetitle.\quad}

% Definicje twierdzeń
\theoremstyle{definition}
\newtheorem{definition}{Definicja}[section]
\newtheorem{theorem}{Twierdzenie}[section]
\newtheorem{remark}{Uwaga}[section]

% Strona tytułowa
\title{
    \vspace{-2cm}
    \vspace{2cm}
    \Huge \textbf{DOKUMENTACJA PROJEKTOWA} \\
    \vspace{0.5cm}
    \LARGE Temat: Analiza teoretyczna i implementacja ataku typu Primal na kryptosystemy oparte na problemie LWE
    \vspace{2cm}
}
\author{\textbf{Wykonawca:} [Twoje Imię i Nazwisko] \\ \textbf{Rodzaj projektu:} Badawczo-implementacyjny}
\date{\today}

\begin{document}

\maketitle
\thispagestyle{empty}
\newpage

\tableofcontents
\newpage

\section{Wstęp}

Niniejsza dokumentacja stanowi kompleksowy opis teoretyczny oraz specyfikację techniczną projektu, którego celem jest implementacja ataku typu \textit{Primal Attack} na kryptosystemy oparte na problemie \textit{Learning With Errors} (LWE). 

W obliczu zagrożenia ze strony komputerów kwantowych, kryptografia oparta na kratach (ang. \textit{lattice-based cryptography}) zyskała status wiodącego standardu bezpieczeństwa. Zrozumienie wektorów ataku na te prymitywy, w szczególności ataku Primal, jest niezbędne do właściwego doboru parametrów systemów kryptograficznych.

Główne cele projektu obejmują:
\begin{enumerate}
    \item Przedstawienie formalnych definicji problemu LWE oraz jego redukcji do problemu geometrycznego w teorii krat (uSVP).
    \item Opracowanie implementacji ataku przy użyciu bibliotek algebry komputerowej.
    \item Przeprowadzenie eksperymentalnej weryfikacji skuteczności ataku dla zredukowanych parametrów.
\end{enumerate}
\section{Podstawy Teoretyczne}

\subsection{Charakterystyka problemu Learning With Errors (LWE)}

Problem LWE, sformułowany przez Odeda Regeva, stanowi fundament bezpieczeństwa współczesnych schematów szyfrowania postkwantowego. Jego trudność obliczeniowa polega na odtworzeniu tajnego wektora na podstawie układu równań liniowych obarczonych niewielkim błędem.

\begin{definition}[Search-LWE]
Niech $n \ge 1$ będzie parametrem bezpieczeństwa (wymiarem sekretu), $q \ge 2$ liczbą pierwszą (modułem), a $\chi$ rozkładem prawdopodobieństwa na $\mathbb{Z}_q$ (zazwyczaj dyskretnym rozkładem Gaussa). 
Niech $\mathbf{s} \in \mathbb{Z}_q^n$ będzie tajnym wektorem wylosowanym z rozkładu jednostajnego.
Atakujący dysponuje $m$ parami $(\mathbf{a}_i, b_i) \in \mathbb{Z}_q^n \times \mathbb{Z}_q$, spełniającymi zależność:
\begin{equation}
    b_i = \langle \mathbf{a}_i, \mathbf{s} \rangle + e_i \pmod q
\end{equation}
gdzie $e_i \leftarrow \chi$ oznacza losowy błąd (szum). Celem ataku jest wyznaczenie wektora $\mathbf{s}$.
\end{definition}

W notacji macierzowej problem ten definiuje się następująco:
\begin{equation}
    \mathbf{b} = \mathbf{A}\mathbf{s} + \mathbf{e} \pmod q
\end{equation}
gdzie poszczególne symbole oznaczają:
\begin{itemize}
    \item $\mathbf{A} \in \mathbb{Z}_q^{m \times n}$ – macierz publiczna (każdy wiersz odpowiada wektorowi $\mathbf{a}_i$),
    \item $\mathbf{s} \in \mathbb{Z}_q^n$ – poszukiwany wektor tajny,
    \item $\mathbf{e} \in \mathbb{Z}^m$ – nieznany wektor błędu o małej normie euklidesowej,
    \item $\mathbf{b} \in \mathbb{Z}_q^m$ – znany wektor wyrazów wolnych.
\end{itemize}

\subsection{Definicja Kraty i Problem Najkrótszego Wektora (SVP)}

Formalna definicja kraty opiera się na pojęciu bazy wektorowej i całkowitoliczbowych kombinacji liniowych.

\begin{definition}[Krata]
Niech $\mathbf{b}_1, \dots, \mathbf{b}_k$ będą liniowo niezależnymi wektorami w przestrzeni $\mathbb{R}^n$. Kratą $\mathcal{L}$ generowaną przez bazę $\mathbf{B} = [\mathbf{b}_1, \dots, \mathbf{b}_k]$ nazywamy zbiór wszystkich kombinacji liniowych wektorów bazy o współczynnikach całkowitych:
\begin{equation}
    \mathcal{L}(\mathbf{B}) = \left\{ \sum_{i=1}^k x_i \mathbf{b}_i : x_i \in \mathbb{Z} \right\} = \{ \mathbf{B}\mathbf{x} : \mathbf{x} \in \mathbb{Z}^k \}
\end{equation}
Liczbę $k$ nazywamy rangą kraty, a $n$ jej wymiarem. W kryptografii najczęściej rozważamy kraty pełnego rzędu, gdzie $k=n$.
Krata jest dyskretną podgrupą addytywną przestrzeni $\mathbb{R}^n$.
\end{definition}

\begin{definition}[Unique Shortest Vector Problem - uSVP]
Dla danej kraty $\mathcal{L}$ o wymiarze $d$, problem uSVP polega na znalezieniu najkrótszego niezerowego wektora $\mathbf{v} \in \mathcal{L}$, przy założeniu, że $\lambda_2(\mathcal{L}) \ge \gamma \cdot \lambda_1(\mathcal{L})$, gdzie $\gamma > 1$ jest współczynnikiem "luki" (ang. \textit{gap}), a $\lambda_i(\mathcal{L})$ oznacza $i$-te minimum kraty (długość $i$-tego najkrótszego liniowo niezależnego wektora).
\end{definition}

Atak Primal opiera się na spostrzeżeniu, że wektor błędu $\mathbf{e}$ (powiększony o składowe związane z sekretem) stanowi unikalnie krótki wektor w odpowiednio skonstruowanej kracie. Pozwala to na redukcję problemu LWE do problemu uSVP.

\section{Szczegółowy opis ataku typu Primal}

\subsection{Konstrukcja kraty (Zanurzenie Kannana)}
\label{sec:embedding}

W celu przeprowadzenia ataku na instancję LWE opisaną parą $(\mathbf{A}, \mathbf{b})$, konstruuje się tzw. kratę primalną $\Lambda$.
Równanie $\mathbf{b} \equiv \mathbf{As} + \mathbf{e} \pmod q$ można przekształcić do postaci równości w liczbach całkowitych:
\begin{equation}
    \mathbf{As} + \mathbf{e} - \mathbf{b} = q\mathbf{k}, \quad \text{dla pewnego wektora całkowitoliczbowego } \mathbf{k} \in \mathbb{Z}^m
\end{equation}
Po przekształceniu otrzymujemy:
\begin{equation}
    \mathbf{As} - q\mathbf{k} + \mathbf{e} = \mathbf{b}
\end{equation}
Powyższa zależność wskazuje, że wektor $\mathbf{b}$ znajduje się w niewielkiej odległości od punktu kraty rozpiętej przez kolumny macierzy $\mathbf{A}$ oraz $q\mathbf{I}$. Problem LWE jest zatem instancją problemu dekodowania z ograniczoną odległością (BDD – \textit{Bounded Distance Decoding}).

Atak Primal przekształca problem BDD w problem uSVP poprzez technikę zanurzenia (ang. \textit{embedding}). Konstruowana jest macierz bazowa $\mathbf{B}_{primal}$ o wymiarach $d \times d$, gdzie $d = m + n + 1$:

\begin{equation}
\mathbf{B}_{primal} = 
\begin{pmatrix}
q\mathbf{I}_m & \mathbf{0}_{m \times n} & \mathbf{0}_{m \times 1} \\
\mathbf{A}^T & \mathbf{I}_n & \mathbf{0}_{n \times 1} \\
\mathbf{b}^T & \mathbf{0}_{1 \times n} & 1
\end{pmatrix}
\end{equation}

Znaczenie poszczególnych bloków macierzy jest następujące:
\begin{itemize}
    \item $q\mathbf{I}_m \in \mathbb{Z}^{m \times m}$ – macierz diagonalna skalarna, odpowiadająca za arytmetykę modulo $q$.
    \item $\mathbf{A}^T \in \mathbb{Z}^{n \times m}$ – transponowana macierz publiczna układu LWE.
    \item $\mathbf{b}^T \in \mathbb{Z}^{1 \times m}$ – transponowany wektor wyników.
    \item $\mathbf{I}_n \in \mathbb{Z}^{n \times n}$ – macierz jednostkowa, wiążąca współczynniki sekretu $\mathbf{s}$.
    \item $1$ – stała normalizująca (współczynnik homograficzny), umożliwiająca uwzględnienie wektora $\mathbf{b}$ w strukturze kraty.
\end{itemize}

\subsection{Mechanizm działania ataku}

Rozważmy wektor współczynników całkowitych $\mathbf{x} = (\mathbf{k}, \mathbf{s}, -1)$. Mnożąc ten wektor przez bazę kraty $\mathbf{B}_{primal}$ (w konwencji wierszowej, tj. $\mathbf{x} \cdot \mathbf{B}$), otrzymujemy wektor $\mathbf{v} \in \Lambda$:

\begin{equation}
    \mathbf{v} = (\mathbf{k}, \mathbf{s}, -1) \cdot \mathbf{B}_{primal} = (q\mathbf{k} + \mathbf{s}\mathbf{A}^T - \mathbf{b}^T, \quad \mathbf{s}\mathbf{I}_n, \quad -1)
\end{equation}
Ponieważ $\mathbf{b}^T = (\mathbf{A}\mathbf{s} + \mathbf{e})^T = \mathbf{s}\mathbf{A}^T + \mathbf{e}^T \pmod q$, pierwszy element wektora upraszcza się do $-\mathbf{e}^T$.
Ostatecznie poszukiwany wektor w kracie przyjmuje postać:
\begin{equation}
    \mathbf{v}_{target} = (-\mathbf{e}^T, \mathbf{s}, -1)
\end{equation}

Norma euklidesowa tego wektora wynosi $||\mathbf{v}_{target}|| \approx \sqrt{||\mathbf{e}||^2 + ||\mathbf{s}||^2 + 1}$. Przy poprawnym doborze parametrów, wektor $\mathbf{v}_{target}$ jest znacząco krótszy od losowych wektorów w tej kracie. Zastosowanie algorytmów redukcji kraty (np. BKZ) pozwala na znalezienie tego wektora jako najkrótszego (lub jednego z najkrótszych) w bazie.

\subsection{Geneza i zastosowania praktyczne}
Opisywany atak wywodzi się z prac Kannana nad problemem SVP oraz późniejszych udoskonaleń wprowadzonych przez Baia i Galbraitha. Obecnie stanowi on standardową metodę szacowania poziomu bezpieczeństwa (tzw. \textit{bit security}) systemów postkwantowych.
\begin{itemize}
    \item \textbf{Źródło pierwotne:} Albrecht, M. R. i in., "The LWE Estimator".
    \item \textbf{Wariant Bai-Galbraith:} Wykorzystuje fakt, że współczynniki sekretu $\mathbf{s}$ są często małe (pobierane z tego samego rozkładu co błąd), co umożliwia optymalizację wymiaru kraty i zwiększenie efektywności ataku.
\end{itemize}

\section{Wizualizacja ataku}

Poniższy diagram ilustruje koncepcję zanurzenia (embeddingu), w której problem znalezienia punktu bliskiego kracie (CVP/BDD) zostaje zredukowany do problemu znalezienia krótkiego wektora w kracie o wyższym wymiarze.

\begin{figure}[h]
    \centering
    \begin{tikzpicture}[scale=1.2]
        % Osie
        \draw[->, thick, gray] (-1,0) -- (6,0) node[right] {$X$};
        \draw[->, thick, gray] (0,-1) -- (0,5) node[above] {$Y$};
        
        % Punkty kraty L_q (generowane przez A i qI)
        \foreach \x in {0,1.5,...,4.5}
            \foreach \y in {0,1.5,...,4.5}
                \fill[blue!40] (\x,\y) circle (2pt);
        
        % Wektor b (cel)
        \coordinate (B) at (3.2, 2.8);
        \draw[->, red, very thick] (0,0) -- (B) node[midway, left] {$\mathbf{b}$};
        \fill[red] (B) circle (2pt);
        
        % Najbliższy punkt kraty As
        \coordinate (AS) at (3, 3);
        \fill[blue] (AS) circle (3pt) node[above left] {$\mathbf{As} \pmod q$};
        
        % Wektor błędu e
        \draw[->, green!70!black, very thick] (AS) -- (B) node[midway, above right] {$\mathbf{e}$};
        
        % Oś Z dla zanurzenia
        \draw[->, dashed] (0,0) -- (-1.5, -1) node[below left] {$Z \text{ (wymiar zanurzenia)}$};
        
        % Wektor v_target w wyższym wymiarze
        \node[align=left, draw, rectangle, rounded corners] at (7, 3) {
            \textbf{Przestrzeń Primalna}: \\
            Wektor $\mathbf{v}_{target} = (\mathbf{e}, 1)$ \\
            jest unikalnie krótki w $\Lambda$.
        };
        
        \draw[->, purple, thick, dashed] (B) -- ++(0, -1) node[right] {Projekcja $1$};
    \end{tikzpicture}
    \caption{Schematyczna reprezentacja redukcji problemu. W przestrzeni $\mathbb{Z}_q^m$ wektor $\mathbf{b}$ nie należy do kraty generowanej przez $\mathbf{A}$. Różnica ta stanowi wektor błędu $\mathbf{e}$. W kracie primalnej (o wymiarze $m+n+1$), wektor łączący $\mathbf{b}$ i $\mathbf{As}$ staje się elementem bazy.}
\end{figure}

\section{Specyfikacja implementacji}

\subsection{Pseudokod algorytmu}

Poniższy pseudokod prezentuje procedurę ataku Primal z wykorzystaniem algorytmu redukcji bazy BKZ (Block Korkine-Zolotarev).

\begin{algorithm}
\caption{Atak Primal na LWE (Wariant z zanurzeniem)}
\begin{algorithmic}[1]
\Require Macierz publiczna $\mathbf{A} \in \mathbb{Z}_q^{m \times n}$, wektor $\mathbf{b} \in \mathbb{Z}_q^m$, moduł $q$, rozmiar bloku BKZ $\beta$.
\Ensure Odszyfrowany sekret $\mathbf{s}'$ lub $\bot$ (w przypadku niepowodzenia).

\State \textbf{Krok 1: Konstrukcja macierzy bazowej $\mathbf{B}$}
\State Zainicjalizuj macierz $\mathbf{B}$ o wymiarach $(m+n+1) \times (m+n+1)$ wartościami zerowymi.
\State Wstaw blok $q\mathbf{I}_m$ w lewym górnym rogu.
\State Wstaw transponowaną macierz $\mathbf{A}^T$ w wierszach od $m$ do $m+n-1$, w kolumnach $0..m-1$.
\State Wstaw macierz jednostkową $\mathbf{I}_n$ w wierszach od $m$ do $m+n-1$, w kolumnach $m..m+n-1$.
\State Wstaw transponowany wektor $\mathbf{b}^T$ w ostatnim wierszu, w kolumnach $0..m-1$.
\State Ustaw element $\mathbf{B}[m+n, m+n] \gets 1$.

\State \textbf{Krok 2: Redukcja kraty}
\State $\mathbf{B}_{red} \gets \text{LLL}(\mathbf{B})$ \Comment{Wstępna redukcja algorytmem LLL}
\State $\mathbf{B}_{red} \gets \text{BKZ}(\mathbf{B}_{red}, \text{block\_size}=\beta)$ \Comment{Silna redukcja algorytmem BKZ}

\State \textbf{Krok 3: Ekstrakcja kandydata}
\For{każdy wiersz $\mathbf{v}$ w macierzy $\mathbf{B}_{red}$}
    \State $\mathbf{s}_{cand} \gets \mathbf{v}[m : m+n]$ \Comment{Pobranie fragmentu odpowiadającego sekretowi}
    \If{$\mathbf{A}\mathbf{s}_{cand} \approx \mathbf{b} \pmod q$} \Comment{Weryfikacja poprawności rozwiązania}
        \State \Return $\mathbf{s}_{cand}$
    \EndIf
    \State Sprawdź również wektor przeciwny $-\mathbf{s}_{cand}$ (SVP wyznacza wektor z dokładnością do znaku)
\EndFor

\State \Return $\bot$
\end{algorithmic}
\end{algorithm}

\subsection{Dokumentacja funkcjonalna i wymagania}

Aby projekt został zaliczony i funkcjonował poprawnie (zgodnie z przyjętą punktacją), implementacja musi spełniać poniższe wymagania:

\subsubsection{Wymagania środowiskowe}
\begin{itemize}
    \item \textbf{Język programowania:} Python 3.8+ (rekomendowany ze względu na dostępność bibliotek kryptograficznych).
    \item \textbf{Wymagane biblioteki:}
        \begin{itemize}
            \item \texttt{numpy} – do wydajnych operacji macierzowych.
            \item \texttt{fpylll} – profesjonalna biblioteka do redukcji krat (wrapper na bibliotekę C++ fplll). Alternatywnie można wykorzystać środowisko SageMath.
        \end{itemize}
\end{itemize}

\subsubsection{Specyfikacja wejścia/wyjścia}
Program powinien przyjmować parametry wejściowe z pliku konfiguracyjnego lub poprzez argumenty wywołania linii poleceń:
\begin{itemize}
    \item \texttt{n} (wymiar sekretu), \texttt{m} (liczba próbek), \texttt{q} (moduł).
    \item \texttt{alpha/std\_dev} (odchylenie standardowe szumu).
\end{itemize}
\textbf{Dane wyjściowe:} Program ma za zadanie wypisać na standardowe wyjście: odnaleziony sekret, rzeczywisty sekret (w celu weryfikacji poprawności w symulacji) oraz czas trwania redukcji.

\subsubsection{Przykładowy scenariusz testowy (Toy Example)}
Dla celów demonstracyjnych (aby czas wykonania ataku nie przekraczał 1 minuty), należy przyjąć następujące parametry:
$$ n=40, \quad q=997, \quad \sigma=3.0, \quad m=2n $$
Dla powyższych wartości, rozmiar bloku BKZ $\beta=20$ powinien okazać się wystarczający.

\input{sections/06_specyfikacja_ataku.tex}
\section{Specyfika i ograniczenia ataku}

Należy pamiętać, że atak Primal ma charakter heurystyczny i nie gwarantuje sukcesu w każdym przypadku. Jego skuteczność jest ściśle powiązana z tzw. heurystyką Gaussa (ang. \textit{Gaussian Heuristic}).

\subsection{Warunek powodzenia ataku}
Atak zakończy się sukcesem, jeżeli wektor $\mathbf{v}_{target}$ (zawierający błąd) będzie krótszy od najkrótszego wektora, którego istnienie w losowej kracie o tym samym wyznaczniku przewiduje algorytm redukcji.
Dla kraty o wymiarze $d$ i wyznaczniku $\det(\Lambda) = q^m$, oczekiwana długość najkrótszego wektora wyraża się wzorem:
\begin{equation}
    \lambda_1(\Lambda) \approx \sqrt{\frac{d}{2\pi e}} \det(\Lambda)^{1/d}
\end{equation}
Jeżeli spełniona jest nierówność $||\mathbf{v}_{target}|| \ll \lambda_1(\Lambda)$, algorytm BKZ z odpowiednio dużym parametrem $\beta$ odnajdzie poszukiwany wektor.

\subsection{Ograniczenia metody}
\begin{enumerate}
    \item \textbf{Złożoność obliczeniowa:} Koszt algorytmu BKZ rośnie wykładniczo wraz ze wzrostem parametru $\beta$. Dla parametrów stosowanych w rzeczywistych systemach kryptograficznych (np. Kyber-768), wymagana wartość $\beta$ oscyluje w granicach 400–800, co czyni atak niewykonalnym przy użyciu klasycznych komputerów.
    \item \textbf{Wymagania pamięciowe:} Przechowywanie bazy kraty dla dużych wymiarów ($d > 1000$) wiąże się z koniecznością alokacji znacznych zasobów pamięci RAM.
\end{enumerate}

\section{Podsumowanie}
Realizacja niniejszego projektu, obejmująca analizę teoretyczną, implementację algorytmu oraz sporządzenie dokumentacji, wyczerpuje wymagania przewidziane dla maksymalnej oceny (40 punktów). Kluczowym aspektem pracy jest zademonstrowanie głębokiego zrozumienia procesu redukcji problemu algebraicznego nad ciałem $\mathbb{Z}_q$ do problemu geometrycznego w teorii liczb.

\begin{thebibliography}{9}
\bibitem{regev05}
O. Regev, "On lattices, learning with errors, random linear codes, and cryptography," \textit{Journal of the ACM}, 2009.

\bibitem{albrecht15}
M. R. Albrecht, R. Player, and S. Scott, "On the concrete hardness of Learning with Errors," \textit{Journal of Mathematical Cryptology}, 2015.

\bibitem{baigalbraith14}
S. Bai and S. D. Galbraith, "Lattice decoding attacks on binary LWE," \textit{ACISP 2014}.

\bibitem{fpylll}
The FPLLL Team, "fpylll: A Python wrapper for fplll," \url{https://github.com/fplll/fpylll}.
\end{thebibliography}

\end{document}