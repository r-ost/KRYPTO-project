\section{Specyfika i ograniczenia ataku}

Należy pamiętać, że atak Primal ma charakter heurystyczny i nie gwarantuje sukcesu w każdym przypadku. Jego skuteczność jest ściśle powiązana z tzw. heurystyką Gaussa (ang. \textit{Gaussian Heuristic}).

\subsection{Warunek powodzenia ataku}
Atak zakończy się sukcesem, jeżeli wektor $\mathbf{v}_{target}$ (zawierający błąd) będzie krótszy od najkrótszego wektora, którego istnienie w losowej kracie o tym samym wyznaczniku przewiduje algorytm redukcji.
Dla kraty o wymiarze $d$ i wyznaczniku $\det(\Lambda) = q^m$, oczekiwana długość najkrótszego wektora wyraża się wzorem:
\begin{equation}
    \lambda_1(\Lambda) \approx \sqrt{\frac{d}{2\pi e}} \det(\Lambda)^{1/d}
\end{equation}
Jeżeli spełniona jest nierówność $||\mathbf{v}_{target}|| \ll \lambda_1(\Lambda)$, algorytm BKZ z odpowiednio dużym parametrem $\beta$ odnajdzie poszukiwany wektor.

\subsection{Ograniczenia metody}
\begin{enumerate}
    \item \textbf{Złożoność obliczeniowa:} Koszt algorytmu BKZ rośnie wykładniczo wraz ze wzrostem parametru $\beta$. Dla parametrów stosowanych w rzeczywistych systemach kryptograficznych (np. Kyber-768), wymagana wartość $\beta$ oscyluje w granicach 400–800, co czyni atak niewykonalnym przy użyciu klasycznych komputerów.
    \item \textbf{Wymagania pamięciowe:} Przechowywanie bazy kraty dla dużych wymiarów ($d > 1000$) wiąże się z koniecznością alokacji znacznych zasobów pamięci RAM.
\end{enumerate}