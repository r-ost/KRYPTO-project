\section{Podstawy Teoretyczne}

\subsection{Charakterystyka problemu Learning With Errors (LWE)}

Problem LWE, sformułowany przez Odeda Regeva, stanowi fundament bezpieczeństwa współczesnych schematów szyfrowania postkwantowego. Jego trudność obliczeniowa polega na odtworzeniu tajnego wektora na podstawie układu równań liniowych obarczonych niewielkim błędem.

\begin{definition}[Search-LWE]
Niech $n \ge 1$ będzie parametrem bezpieczeństwa (wymiarem sekretu), $q \ge 2$ liczbą pierwszą (modułem), a $\chi$ rozkładem prawdopodobieństwa na $\mathbb{Z}_q$ (zazwyczaj dyskretnym rozkładem Gaussa). 
Niech $\mathbf{s} \in \mathbb{Z}_q^n$ będzie tajnym wektorem wylosowanym z rozkładu jednostajnego.
Atakujący dysponuje $m$ parami $(\mathbf{a}_i, b_i) \in \mathbb{Z}_q^n \times \mathbb{Z}_q$, spełniającymi zależność:
\begin{equation}
    b_i = \langle \mathbf{a}_i, \mathbf{s} \rangle + e_i \pmod q
\end{equation}
gdzie $e_i \leftarrow \chi$ oznacza losowy błąd (szum). Celem ataku jest wyznaczenie wektora $\mathbf{s}$.
\end{definition}

W notacji macierzowej problem ten definiuje się następująco:
\begin{equation}
    \mathbf{b} = \mathbf{A}\mathbf{s} + \mathbf{e} \pmod q
\end{equation}
gdzie poszczególne symbole oznaczają:
\begin{itemize}
    \item $\mathbf{A} \in \mathbb{Z}_q^{m \times n}$ – macierz publiczna (każdy wiersz odpowiada wektorowi $\mathbf{a}_i$),
    \item $\mathbf{s} \in \mathbb{Z}_q^n$ – poszukiwany wektor tajny,
    \item $\mathbf{e} \in \mathbb{Z}^m$ – nieznany wektor błędu o małej normie euklidesowej,
    \item $\mathbf{b} \in \mathbb{Z}_q^m$ – znany wektor wyrazów wolnych.
\end{itemize}

\subsection{Definicja Kraty i Problem Najkrótszego Wektora (SVP)}

Formalna definicja kraty opiera się na pojęciu bazy wektorowej i całkowitoliczbowych kombinacji liniowych.

\begin{definition}[Krata]
Niech $\mathbf{b}_1, \dots, \mathbf{b}_k$ będą liniowo niezależnymi wektorami w przestrzeni $\mathbb{R}^n$. Kratą $\mathcal{L}$ generowaną przez bazę $\mathbf{B} = [\mathbf{b}_1, \dots, \mathbf{b}_k]$ nazywamy zbiór wszystkich kombinacji liniowych wektorów bazy o współczynnikach całkowitych:
\begin{equation}
    \mathcal{L}(\mathbf{B}) = \left\{ \sum_{i=1}^k x_i \mathbf{b}_i : x_i \in \mathbb{Z} \right\} = \{ \mathbf{B}\mathbf{x} : \mathbf{x} \in \mathbb{Z}^k \}
\end{equation}
Liczbę $k$ nazywamy rangą kraty, a $n$ jej wymiarem. W kryptografii najczęściej rozważamy kraty pełnego rzędu, gdzie $k=n$.
Krata jest dyskretną podgrupą addytywną przestrzeni $\mathbb{R}^n$.
\end{definition}

\begin{definition}[Unique Shortest Vector Problem - uSVP]
Dla danej kraty $\mathcal{L}$ o wymiarze $d$, problem uSVP polega na znalezieniu najkrótszego niezerowego wektora $\mathbf{v} \in \mathcal{L}$, przy założeniu, że $\lambda_2(\mathcal{L}) \ge \gamma \cdot \lambda_1(\mathcal{L})$, gdzie $\gamma > 1$ jest współczynnikiem "luki" (ang. \textit{gap}), a $\lambda_i(\mathcal{L})$ oznacza $i$-te minimum kraty (długość $i$-tego najkrótszego liniowo niezależnego wektora).
\end{definition}

Atak Primal opiera się na spostrzeżeniu, że wektor błędu $\mathbf{e}$ (powiększony o składowe związane z sekretem) stanowi unikalnie krótki wektor w odpowiednio skonstruowanej kracie. Pozwala to na redukcję problemu LWE do problemu uSVP.
