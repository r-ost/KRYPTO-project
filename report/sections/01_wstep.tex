\section{Wstęp}

Niniejsza dokumentacja stanowi kompleksowy opis teoretyczny oraz specyfikację techniczną projektu, którego celem jest implementacja ataku typu \textit{Primal Attack} na kryptosystemy oparte na problemie \textit{Learning With Errors} (LWE). 

W obliczu zagrożenia ze strony komputerów kwantowych, kryptografia oparta na kratach (ang. \textit{lattice-based cryptography}) zyskała status wiodącego standardu bezpieczeństwa. Zrozumienie wektorów ataku na te prymitywy, w szczególności ataku Primal, jest niezbędne do właściwego doboru parametrów systemów kryptograficznych.

Główne cele projektu obejmują:
\begin{enumerate}
    \item Przedstawienie formalnych definicji problemu LWE oraz jego redukcji do problemu geometrycznego w teorii krat (uSVP).
    \item Opracowanie implementacji ataku przy użyciu bibliotek algebry komputerowej.
    \item Przeprowadzenie eksperymentalnej weryfikacji skuteczności ataku dla zredukowanych parametrów.
\end{enumerate}